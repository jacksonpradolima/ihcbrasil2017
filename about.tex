\documentclass{sigchi}

%configuração de Copyright
%% Use this section to set the ACM copyright statement (e.g. for
% preprints).  Consult the conference website for the camera-ready
% copyright statement.

% Copyright
\CopyrightYear{2017}
%\setcopyright{acmcopyright}
\setcopyright{acmlicensed}
%\setcopyright{rightsretained}
%\setcopyright{usgov}
%\setcopyright{usgovmixed}
%\setcopyright{cagov}
%\setcopyright{cagovmixed}
% DOI
\doi{http://dx.doi.org/XX.XXX/XXX_X}
% ISBN
\isbn{XXX-XXXX-X-567/08/06}
%Conference
\conferenceinfo{CHI'16,}{May 07--12, 2016, San Jose, CA, USA}
%Price
\acmPrice{\$15.00}

% Use this command to override the default ACM copyright statement
% (e.g. for preprints).  Consult the conference website for the
% camera-ready copyright statement.

%% HOW TO OVERRIDE THE DEFAULT COPYRIGHT STRIP --
%% Please note you need to make sure the copy for your specific
%% license is used here!
% \toappear{
% Permission to make digital or hard copies of all or part of this work
% for personal or classroom use is granted without fee provided that
% copies are not made or distributed for profit or commercial advantage
% and that copies bear this notice and the full citation on the first
% page. Copyrights for components of this work owned by others than ACM
% must be honored. Abstracting with credit is permitted. To copy
% otherwise, or republish, to post on servers or to redistribute to
% lists, requires prior specific permission and/or a fee. Request
% permissions from \href{mailto:Permissions@acm.org}{Permissions@acm.org}. \\
% \emph{CHI '16},  May 07--12, 2016, San Jose, CA, USA \\
% ACM xxx-x-xxxx-xxxx-x/xx/xx\ldots \$15.00 \\
% DOI: \url{http://dx.doi.org/xx.xxxx/xxxxxxx.xxxxxxx}
% }

% configurações de diversos pacotes da ACM
% formato dos arquivos-fonte (utf8 no Linux e latin1 no Windows)
\usepackage[utf8]{inputenc}	% arquivos LaTeX em Unicode (UTF8)

% usar codificação T1 para ter caracteres acentuados corretos no PDF
\usepackage[T1]{fontenc}

% Arabic page numbers for submission.  Remove this line to eliminate
% page numbers for the camera ready copy
% \pagenumbering{arabic}

% Load basic packages
\usepackage{balance}       % to better equalize the last page
\usepackage{graphics}      % for EPS, load graphicx instead 
\usepackage[T1]{fontenc}   % for umlauts and other diaeresis
\usepackage{txfonts}
\usepackage{mathptmx}
\usepackage[pdflang={en-US},pdftex]{hyperref}
\usepackage{color}
\usepackage{booktabs}
\usepackage{textcomp}

% Some optional stuff you might like/need.
\usepackage{microtype}        % Improved Tracking and Kerning
% \usepackage[all]{hypcap}    % Fixes bug in hyperref caption linking
\usepackage{ccicons}          % Cite your images correctly!
% \usepackage[utf8]{inputenc} % for a UTF8 editor only

% If you want to use todo notes, marginpars etc. during creation of
% your draft document, you have to enable the "chi_draft" option for
% the document class. To do this, change the very first line to:
% "\documentclass[chi_draft]{sigchi}". You can then place todo notes
% by using the "\todo{...}"  command. Make sure to disable the draft
% option again before submitting your final document.
\usepackage{todonotes}

% Paper metadata (use plain text, for PDF inclusion and later
% re-using, if desired).  Use \emtpyauthor when submitting for review
% so you remain anonymous.
\def\plaintitle{SIGCHI Conference Proceedings Format}
\def\plainauthor{Jackson A. Prado Lima, Thiago [sobrenome?], Rodrigo Mello, José Ramalho, Guilherme Cintra}
\def\emptyauthor{}
\def\plainkeywords{Authors' choice; of terms; separated; by
  semicolons; include commas, within terms only; required.}
\def\plaingeneralterms{Documentation, Standardization}

% llt: Define a global style for URLs, rather that the default one
\makeatletter
\def\url@leostyle{%
  \@ifundefined{selectfont}{
    \def\UrlFont{\sf}
  }{
    \def\UrlFont{\small\bf\ttfamily}
  }}
\makeatother
\urlstyle{leo}

% To make various LaTeX processors do the right thing with page size.
\def\pprw{8.5in}
\def\pprh{11in}
\special{papersize=\pprw,\pprh}
\setlength{\paperwidth}{\pprw}
\setlength{\paperheight}{\pprh}
\setlength{\pdfpagewidth}{\pprw}
\setlength{\pdfpageheight}{\pprh}

% Make sure hyperref comes last of your loaded packages, to give it a
% fighting chance of not being over-written, since its job is to
% redefine many LaTeX commands.
\definecolor{linkColor}{RGB}{6,125,233}
\hypersetup{%
  pdftitle={\plaintitle},
% Use \plainauthor for final version.
%  pdfauthor={\plainauthor},
  pdfauthor={\emptyauthor},
  pdfkeywords={\plainkeywords},
  pdfdisplaydoctitle=true, % For Accessibility
  bookmarksnumbered,
  pdfstartview={FitH},
  colorlinks,
  citecolor=black,
  filecolor=black,
  linkcolor=black,
  urlcolor=linkColor,
  breaklinks=true,
  hypertexnames=false
}

% create a shortcut to typeset table headings
% \newcommand\tabhead[1]{\small\textbf{#1}}

% End of preamble. Here it comes the document.
\begin{document}

\title{\plaintitle}

\numberofauthors{5}
\author{%
  \alignauthor{Jackson A. Prado Lima\\
    \affaddr{Department of Computer Science, Federal University of Paraná}\\
    \affaddr{Curitiba, Brazil}\\
    \email{japlima@inf.ufpr.br}}\\
  \alignauthor{Thiago\\
    \affaddr{for Submission}\\
    \affaddr{Curitiba, Brazil}\\
    \email{e-mail address}}\\
  \alignauthor{Rodrigo Mello\\
    \affaddr{for Submission}\\
    \affaddr{Curitiba, Brazil}\\
    \email{e-mail address}}\\  
  \alignauthor{José Ramalho\\
    \affaddr{for Submission}\\
    \affaddr{Curitiba, Brazil}\\
    \email{e-mail address}}\\
  \alignauthor{Guilherme Luiz Cintra Neves\\
    \affaddr{Library of Human Sciences, Library System, Federal University of Paraná}\\
    \affaddr{Curitiba, Brazil}\\
    \email{cintra@ufpr.br}}\\  
}

\maketitle

%=====================================================

% corpo do documento

\section{Sobre o trabalho}

Aqui está escrito uma "base" de informações sobre o trabalho.

De acordo com a proposta de trabalho fornecida pela organização da competição de Design o tema abrange educadores e/ou crianças de 0-10 anos. Dessa forma, devemos determinar um público alvo para dar um devido tratamento e auxílio à esse público. Para isso, foi determinado como público alvo crianças de 4 a 5 anos. Nessa faixa de idade as crianças estão entrando na alfabetização, mas ainda não são alfabetizadas. 

De modo a entender melhor a questão de alfabetização é necessário conceiturar primeiramente letramento. Letramento refere-se ao processo que envolve objetivos tais como a inserção da criança na cultura escrita, a criação de possibilidades para a participação da criança em experiências variadas com a escrita e a leitura, o conhecimento e o reconhecimento dos diversos gêneros textuais. Por outro lado, de acordo com Galvão e Leal~\cite{galvao2005ha}, a alfabetização é um processo de construção de hipóteses sobre o funcionamento do sistema alfabético de escrita. Para aprender a ler e a escrever, o aluno precisa participar de situações que o desafiem, que coloquem a necessidade da reflexão sobre a língua, que o leve enfim a transformar informações em conhecimento próprio. É utilizando-se de textos reais, tais como listas, poemas, bilhetes, receitas, contos, piadas, entre outros gêneros, que os alunos podem aprender muito sobre a escrita (\cite{galvao2005ha}).

Dominar a língua é uma forma de dominar uma determinada realidade social e cultural. Segundo Soares~\cite{soares2002letramento}, um indivíduo alfabetizado não é necessariamente um indivíduo letrado; a alfabetização é aquele que sabe ler e escrever; já o indivíduo letrado, o indivíduo que vive em estado de letramento, é não só aquele que sabe ler e escrever, mas aquele que usa socialmente a leitura e a escrita, pratica a leitura e a escrita, responde adequadamente às demandas sociais de leitura e de escrita.

É necessário perceber que as crianças aprendem de formas diferentes, sendo isso um desafio aos educadores. Desse modo, como já mencionado anteriormente, o presente trabalho possui enfoque na alfabetiação de crianças de 4 a 5 anos. Segundo 




Fase de desenvolvimento de Piaget - pré-operatório, Passos de silabação (tem níveis) do método sintético, vai silábico sintético até ser alfabetizada.

Fases do desenvolvimento: Piaget
Emilia Ferreira e Ana teberoski.
Já existem jogos de alfabetização em sites. Usado como uma ferramenta de auxílio. Esses jogos são muitos estáticos, deveriam tbm aprender com os alunos.
É sempre a mesma coisa. A primeira vez é um desavio, depois ela não precisa pensar.
Clica na silaba ela fala. Formar bola... bo fala bo... etc
Imagem confundiu. Telefone e celular.

Jogos silábicos formar palavras, reconhecimento de letras. Não foi visto jogos com músicas. 
Música prende mais atenção nas crianças.
Formar a música atirei o pau no gato, formando letras. Poemas... não partir pra música pq é complexo.
Seria interessante formar essas palavras dentro de uma música. Tipo karaokê?
Fonética.
4 anos já tem que estar na escola. É lei.

O André (filho do Carlos) leva tablet, a alternativa “exige”. Celular deveria ser complementar.

Tem que ter muita cor. Cuidar com as cores por causa de ataques – Pokemon
Coisas atrativas. 

Jogo que não evolui faz a criança cansar, perder o foco.

Ao avançar de nível, fazer a criança não perder o foco por ser repetitivo.

Pepa tem cor, fala linguagem da criança. Tem momento de interação, tipo a dora, pergunta pra criança coisas.

Introdução em sala pra usar o lab de informática.

Ter uma história pra ensinar vogais, etc. Em desenho pras crianças entenderem. Ter algo que professor e software interajam pra complementar. Ter um mediador
Tipo faculdade ead. Videos curtos.

Alfabetização com textos, músicas e poema acho que não tem.

Começar com um repertório pequeno de letras e ir aumentando.

Criança confunde com o método caonico (acho que é isso). Bola, casa e quando chega em calha a criança confunde. Muita vogal consoante juntas...

O educador cuida das crianças especiais. Tipo, ela tem dez mas desenvolvimento de 5.

Se for musical ajuda cego.

Pra surdo, um visual de certo, errado pra entender.

\balance{}

% REFERENCES FORMAT
% References must be the same font size as other body text.
\bibliographystyle{SIGCHI-Reference-Format}
\bibliography{references}

\end{document}