\documentclass{sigchi}

%configuração de Copyright
%\include{copyright}

% configurações de diversos pacotes da ACM
\include{packages}

% End of preamble. Here it comes the document.
\begin{document}

\title{\plaintitle}

\numberofauthors{5}
\author{%
  \alignauthor{Jackson A. Prado Lima\\
    \affaddr{Department of Computer Science, Federal University of Paraná}\\
    \affaddr{Curitiba, Brazil}\\
    \email{japlima@inf.ufpr.br}}\\
  \alignauthor{Thiago\\
    \affaddr{for Submission}\\
    \affaddr{Curitiba, Brazil}\\
    \email{e-mail address}}\\
  \alignauthor{Rodrigo Mello\\
    \affaddr{for Submission}\\
    \affaddr{Curitiba, Brazil}\\
    \email{e-mail address}}\\  
  \alignauthor{José Ramalho\\
    \affaddr{for Submission}\\
    \affaddr{Curitiba, Brazil}\\
    \email{e-mail address}}\\
  \alignauthor{Guilherme L. Cintra Neves\\
    \affaddr{Library of Human Sciences, Federal University of Paraná}\\
    \affaddr{Curitiba, Brazil}\\
    \email{cintra@ufpr.br}}\\  
}

\maketitle

%=====================================================

% corpo do documento

\section{Sobre o trabalho}

De acordo com a proposta de trabalho fornecida pela organização da competição de Design o tema abrange educadores e/ou crianças de 0-10 anos. Dessa forma, devemos determinar um público alvo para dar um devido tratamento e auxílio à esse público. Para isso, foi determinado como público alvo crianças de 4 a 5 anos, em que possui crianças que estão entrando na alfabetização. 

De modo a entender melhor a questão de alfabetização é necessário conceituar primeiramente letramento. Letramento refere-se ao processo que envolve objetivos tais como a inserção da criança na cultura escrita, a criação de possibilidades para a participação da criança em experiências variadas com a escrita e a leitura, o conhecimento e o reconhecimento dos diversos gêneros textuais. Por outro lado, de acordo com Galvão e Leal~\cite{galvao2005ha}, a alfabetização é um processo de construção de hipóteses sobre o funcionamento do sistema alfabético de escrita. Para aprender a ler e a escrever, o aluno precisa participar de situações que o desafiem, que coloquem a necessidade da reflexão sobre a língua, que o leve enfim a transformar informações em conhecimento próprio. É utilizando-se de textos reais, tais como listas, poemas, bilhetes, receitas, contos, piadas, entre outros gêneros, que os alunos podem aprender muito sobre a escrita (Galvão e Leal~\cite{galvao2005ha}).

Dominar a língua é uma forma de dominar uma determinada realidade social e cultural. Segundo Soares~\cite{soares2002letramento}, um indivíduo alfabetizado não é necessariamente um indivíduo letrado; a alfabetização é aquele que sabe ler e escrever; já o indivíduo letrado, o indivíduo que vive em estado de letramento, é não só aquele que sabe ler e escrever, mas aquele que usa socialmente a leitura e a escrita, pratica a leitura e a escrita, responde adequadamente às demandas sociais de leitura e de escrita.

É necessário perceber que as crianças aprendem de formas diferentes, sendo isso um desafio aos educadores. Desse modo, como já mencionado anteriormente, o presente trabalho possui enfoque na alfabetização de crianças de 4 a 5 anos. Essa faixa etária está inserida no estágio simbólico segundo Piaget~\cite{piaget1994psicologia}, 2 a 7 anos, conhecida também como segundo período pré-operatório. Nesse estágio, o pensamento da criança está centrado nela mesma, um pensamento egocêntrico, sendo nesta fase em que se apresenta a linguagem, como socialização da criança, que se dá através da fala, dos desenhos e das dramatizações. 

De acordo com Ferreiro et \textit{al.}~\cite{ferreiro1986psicogenese}, toda criança passa por quatro fases até sua alfabetização:

\begin{itemize}
    \item pré-silábica: não consegue relacionar as letras com os sons da língua falada;
    \item silábica: interpreta de sua maneira, atribuindo valor a cada sílaba;
    \item silábico-alfabética: mistura a lógica da fase anterior com a identificação de cada silaba;
    \item alfabética: domina o valor das letras e sílabas.
\end{itemize}
    
O processo de conhecimento da criança deve ser gradual dependendo de sua assimilação e de uma re-acomodação dos esquemas internos, que necessariamente levam tempo. É por utilizar esse sistema e não repetir o que ouvem, que as crianças interpretam o ensino que recebem. Nada mais revelador do funcionamento da mente de um aluno do que seus supostos erros, porque evidenciam como ele releu o conteúdo aprendido.

Segundo Ferreiro~\cite{ferreiro1996alfabetizacao} “O desenvolvimento da alfabetização ocorre, sem dúvida, em um ambiente social. Mas as práticas sociais assim como as informações sociais, não são recebidas passivamente pelas crianças.”

Desse modo, através da utilização do método sintético, analítico e misto, o presente trabalho busca auxiliar a alfabetização e letramento da criança. 

Devido ao presente trabalho se enquadrar como ``jogo'' foi realizado uma revisão dos jogos educacionais mais utilizados. Nesse contexto, esses jogos são utilizados por educadores como uma ferramenta de auxílio. Entretanto, tais jogos são considerados muito estáticos, em que o conhecimento da criança não é utilizado como informação ao jogo e assim definindo níveis mais difíceis ou mais fáceis de acordo com o aprendizado da criança, ou seja, o jogo também deve aprender com a criança. Os jogos inicialmente geram um desafio para a criança, de modo que a criança necessite pensar e não apenas repetir, porém a repetitividade do jogo faz com que esse desafio torne-se tedioso e tirando o foco da criança com outras coisas.

Para isso, o presente trabalho busca: reconhecimento de letras, sílabas e palavras com o uso de música para que a criança preste mais atenção e utilize a musicoterapia como auxílio de aprendizado. Utilizando imagens de apoio que não confundam a criança, por exemplo, celular e telefone. Começar com um repertório pequeno de letras e ir aumentando. Posteriormente, a criança deverá formar palavras, em seguida, frases e avançando para poemas, rimas, músicas completas, etc. Na formação de textos maiores, como frases, poemas e músicas, o jogo ajuda para a criança acompanhar a formação como um Karaokê, formando a palavra dentro de uma música e cantando com ela. Além disso, entre as atividades há contações de histórias de pouca duração para a criança, sendo que são partes que se complementam, assim, a criança receberá uma recompensa pela finalização da atividade. Nas transições de níveis um filme educativo de pouca duração é exibido à criança. Desse modo, a criança permanece com o foco junto ao jogo. 
% * <Jackson Antonio do Prado Lima> 14:48:59 20 Mar 2017 UTC-0300:
% Creio que se fazermos um diagrama ficaria mais entendível

De modo a chamar a atenção da criança o uso de muita cor com visualizações atrativas e uso de vídeo são pontos chave do trabalho. Além das tratativas para evitar possíveis problemas de saúde relacionados como o esquema de cores, por exemplo, convulsões.

Os dados obtidos pela aprendizagem das crianças podem ser utilizados para futuras análises. O jogo possibilita que crianças de outras idades possam utilizar, até mesmo aquelas de idade avançada com desenvolvimento inferior ao da idade. O uso de sons permite que pessoas com deficiência visual utilizem. Para as pessoas com problemas de audição o uso de imagens, seja para mostrar se acertou ou errou, possibilita a interação destas com o jogo.

\balance{}

% REFERENCES FORMAT
% References must be the same font size as other body text.
\bibliographystyle{SIGCHI-Reference-Format}
\bibliography{references}

\end{document}