\documentclass{sigchi}

%configuração de Copyright
%\include{copyright}

% configurações de diversos pacotes da ACM
\include{packages}

% End of preamble. Here it comes the document.
\begin{document}

\title{\plaintitle}

\numberofauthors{5}
\author{%
  \alignauthor{Jackson A. Prado Lima\\
    \affaddr{Department of Computer Science, Federal University of Paraná}\\
    \affaddr{Curitiba, Brazil}\\
    \email{japlima@inf.ufpr.br}}\\
  \alignauthor{Thiago\\
    \affaddr{for Submission}\\
    \affaddr{Curitiba, Brazil}\\
    \email{e-mail address}}\\
  \alignauthor{Rodrigo Mello\\
    \affaddr{for Submission}\\
    \affaddr{Curitiba, Brazil}\\
    \email{e-mail address}}\\  
  \alignauthor{José Ramalho\\
    \affaddr{for Submission}\\
    \affaddr{Curitiba, Brazil}\\
    \email{e-mail address}}\\
  \alignauthor{Guilherme Luiz Cintra Neves\\
    \affaddr{Library of Human Sciences, Library System, Federal University of Paraná}\\
    \affaddr{Curitiba, Brazil}\\
    \email{cintra@ufpr.br}}\\  
}

\maketitle

\begin{abstract}
  This sample paper describes the formatting
  requirements for SIGCHI conference proceedings, and offers
  recommendations on writing for the worldwide SIGCHI
  readership. Please review this document even if you have submitted
  to SIGCHI conferences before, as some format details have changed
  relative to previous years. Abstracts should be about 150 words and
  are required.
\end{abstract}

\category{H.5.m.}{Information Interfaces and Presentation
  (e.g. HCI)}{Miscellaneous} \category{See
  \url{http://acm.org/about/class/1998/} for the full list of ACM
  classifiers. This section is required.}{}{}

\keywords{\plainkeywords}

%=====================================================

% corpo do documento

\section{Introduction}

Contexto e problema abordado


 Contexto: Crianças em fase pré escolar - 6o estágio de desenvolviment cognitivo (18-24 meses) - Piaget; Musicoterapia.
\section{Methodology}

quais foram os caminhos seguidos para encontrar a solução (métodos empregados, pesquisa de campo, etc)
\section{Justification}

por quê a solução proposta é relevante ao contexto, por quê os métodos usados são adequados
\section{Proposed Method}

Solução proposta e Cenário de uso (com link para mockup interativo): descrição de um cenário de uso
\section{Viability}

fazer uma argumentação crítica, pontuando se a solução pode ser desenvolvida com os recursos atuais da tecnologia

%Adicionar link para video demo
% * <Jackson Antonio do Prado Lima> 13:50:15 17 Mar 2017 UTC-0300:
% Lembrar

\section{Acknowledgments}

This work is supported by Brazilian funding agencies CAPES and CNPq. Grants: XXXXXX/XXXX-X and XXXXXX/XXXX-X.

% Balancing columns in a ref list is a bit of a pain because you
% either use a hack like flushend or balance, or manually insert
% a column break.  http://www.tex.ac.uk/cgi-bin/texfaq2html?label=balance
% multicols doesn't work because we're already in two-column mode,
% and flushend isn't awesome, so I choose balance.  See this
% for more info: http://cs.brown.edu/system/software/latex/doc/balance.pdf
%
% Note that in a perfect world balance wants to be in the first
% column of the last page.
%
% If balance doesn't work for you, you can remove that and
% hard-code a column break into the bbl file right before you
% submit:
%
% http://stackoverflow.com/questions/2149854/how-to-manually-equalize-columns-
% in-an-ieee-paper-if-using-bibtex
%
% Or, just remove \balance and give up on balancing the last page.
%
\balance{}

% REFERENCES FORMAT
% References must be the same font size as other body text.
\bibliographystyle{SIGCHI-Reference-Format}
\bibliography{references}

\end{document}